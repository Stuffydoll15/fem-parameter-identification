\begin{abstract}
\addchaptertocentry{\abstractname} % Add the abstract to the table of contents
This study proposes a framework for the identification of soft material parameters, 
focusing on those materials displaying nonlinear responses, 
through an inverse Finite Element Method (iFEM) process. This process integrates experimental and computational models,
utilizing load-displacement data collected from the indentation experiment conducted on 
a representative ultra-soft polyurethane sample, to establish a 
computational model that mimics the indentation experiment. Furthermore, this computational model 
overcomes the challenge of element high distortion effect during simulations using the nonlinear adaptive region meshing.\\

Within this framework, material models ranging from linear elasticity to Neo-Hookean were explored, 
with key parameters identified and their interrelationships analyzed. Optimization was achieved through the 
application of a Response Surface Optimization (RSO) algorithm, utilizing a performance metric defined by the disparity 
between experimental and simulated outcomes. The robustness and limitations of this approach were evaluated through three 
validation cases.\\

The study also emphasizes opportunities for enhancing the framework's applicability, such as multi-point 
indentation, reevaluation of experimental indentation point selection for iFEM, further constraint analysis, consideration of different 
indentation directions for organ applications, and exploration of alternative hyperelastic models. Collaborations with 
field experts to establish an error metric threshold were recommended for biomaterial studies.\\

The primary contribution of this research lies in the development of a systematic hierarchy for material models,
 enhancing the precision of material model response while maintaining model simplicity. Recognizing its current limitations, 
 the framework presents a foundation for future research, particularly in the mechanical characterization of 
 organs for \textit{ex vivo} experiments. Therefore, this study pursues the development of simple yet accurate 
 and efficient material models for complex materials, advancing the active field of soft tissue mechanical characterization.
\end{abstract}
