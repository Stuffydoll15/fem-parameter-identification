% Chapter 3

\chapter{Computational model} % Main chapter title
\label{chapter:computationalmodel} % For referencing the chapter elsewhere, use \ref{Chapter1} 

In the previous chapter the experimental model designs for the iFEM process of the material 
parameter identification were illustrated. The experimental setups, the results, and the main assumptions 
made for the material models laid the foundation for the development of the finite element (FE) model.

In this chapter, the computational models of the indentation tests described in Chapter \ref{chapter:experimentalmodel} 
will be described. These models were built using the finite element software ANSYS. 
The computational model serves as principal link between the experimental data and identification of the 
specimen's material parameters.  

The initial simulation model, referred to as Computational model I, served as a tool 
for testing the preliminary ideas, as well as establishing the basis structure for 
the computational model. This first step was essential to identify the strengths and 
weaknesses of the model, and it provided insights that guided the development of an 
improved simulation model, referred to as Computational model II. By exploring the 
results and challenges of computational model I, the initial ideas 
were iterated and a more robust and accurate computational model was created for the material
parameters identification process.

In this chapter the development process of the computational models will be detailed, 
as well as the encountered complications and their proposed solutions, 
and the improvements made in computational model II.

%-----------------------------------------------------------------------------------------
\section{Computational model I (CP I)}
\subsection{Middle point}
The quasi static nature of the indentation experiment allowed the use of a static 
structural analysis. As the most important factor is the development of the framework for identification 
of the material parameter process, the model was simplified as much as possible to avoid the influence 
of the error produced by complex geometries and the definition of the boundary conditions.

\subsubsection*{Description}

For the geometry, only the key components that significantly contribute to the mechanical response were modeled, 
i.e., the indenter and the specimen. The tumor extraction geometry shown in Figure \ref{fig:specimenhole}
was also simulated for CP I. Both, the indenter and the specimen used SOLID187 elements. 
SOLID187 elements are a high order 3-D, 10 node elements with three degrees of freedom at each node. 
This element was suitable to simulate the deformations of nearly incompressible hyperelastic materials 
and had the hyperelasticity, large deflection, and large strain capabilities \cite{Ansys2010}.\\

The geometry model used for EP I is observed in Figure. %figure of the middle point geometry for experimental model I and II
This figure shows the FE model for the middle point test configuration, even though symmetry can be applied, this was 
not employed. The main reason for this was because future project steps were taken into consideration. 
For future steps in this project the measurement and identification of human organs material parameter 
is the desired goal. In this case, due to the anisotropy of the material and the geometry complexity,
symmetry boundary conditions cannot be applied. 


which allowed the used 
of symmetry boundary conditions. This characteristic reduced the computational cost for the simulation 
of the FE model, therefore one-quarter model was used for EP I.\\

%Maybe create table of the initial parameters chosen? which part is this going to be?
To approximate the mechanical properties of ultra-soft polyurethane, initial parameters were assigned 
to the computational model. For the linear elasticity level, values for the Young's modulus $E$ and the 
Poisson's ratio $\nu$ were assigned. At the hyperelastic level, the initial parameters for the Neo-Hookean model, 
the shear modulus $\mu$ and the incompressibility parameter $D_1$ were specified.\\ 

The contact interaction between the indenter and the specimen were set frictionless, to reduce 
overall complexity of the model and mimic the lubricated contact in the experiments.
The lower half of the specimen was defined as a fixed support, reflecting the physical constraint in the 
experimental setup. The displacement value was applied to the indenter in Y-direction, 
from which the force reaction was obtained.
%CONTA174 and TARGE170

The mesh for the whole model were formed from quadratic tetrahedral elements. 
Various meshing methods were tried and the mesh, which showed the best results for experimental model I 
possessed an global element size of $e_{I_s}=\SI{5}{\milli\meter}$ and $e_{I_i}=\SI{1}{\milli\meter}$ for the specimen 
and the indenter respectively. A refinement was made in the area of interest, i.e., the contact 
area between the indenter and the specimen's surface, with a sphere radius of 
$r_{I_a}=\SI{8}{\milli\meter}$ and an element size of $e_{I_a}=\SI{0.5}{\milli\meter}$ (Fig.). %add meshing figure.

\subsubsection*{Results}

\section{Analysis and Complications}
Their a two main factors which increases the complexity of the validation of the simulation
and those are, the contact nonlinearity, and the element distortion due to indentation
 experiment. These issues make the computational time expensive, as it requires to manual 
 solutions for the meshing in the area of importance, and small time steps. 
 For that, 
 the nonlinear adaptive meshing option in ANSYS Workbench was applied, which does a remeshing
 process if the a certain parameter is exceeded.%Revisar explanation of nonlinear adaprive meshing
Specially, for larger indentation cases, this option shows a more stable model with a 
good mesh convergence analysis.

A force-displacement curve, shown in Fig... is generated from the first assumption, 
for this case 
%Additionally due to these factors, the converge analysis   

\section*{Experimental model II}
This was removed for the EP II geometry model, as it was shown that this area 
of the specimen's geometry was irrelevant for the calculation of the results.
For EP I and EP II only one-quarter, and the half of the full model was used respectively.
The decision to use half model for EP II was made to make the visualization of the deformation profile 
easier for the validation in further steps.\\

The mesh for the whole model is formed from quad tetrahedral elemennts. The platform and 
the specimen have an global element size of 5 mm. The indenter has an element size of
0.5 mm. In the area of the indentation, there is finer mesh with an element size of
1 mm and a radius of 8 mm. 



For both cases 

\subsection{Verification of the Simulation Model}

\subsubsection*{Mesh Convergence Analysis}
\subsubsection*{Platform vs Fixed Support}


%----------------------------------------------------------------------------------
\section{Nearby point}


