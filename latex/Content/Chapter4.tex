% Chapter 3

\chapter{Computational model} % Main chapter title

\label{Chapter4} % For referencing the chapter elsewhere, use \ref{Chapter1} 

\section{Middle point}
\subsection{Description}
asas
The quasi static nature of the indentation experiment allows the use of a static 
structural analysis.

For the creation of the computational model, the SOLID 187 elements were used. The mesh 
for the whole model is formed from quad tetrahedral elemennts. The platform and 
the specimen have an global element size of 5 mm. The indenter has an element size of
0.5 mm. In the area of the indentation, there is finer mesh with an element size of
1 mm and a radius of 8 mm. 

\subsection{Analysis and Complications}
Their a two main factors which increases the complexity of the validation of the simulation
and those are, the contact nonlinearity, and the element distortion due to indentation
 experiment. These issues make the computational time expensive, as it requires to manual 
 solutions for the meshing in the area of importance, and small time steps. 
 For that, 
 the nonlinear adaptive meshing option in ANSYS Workbench was applied, which does a remeshing
 process if the a certain parameter is exceeded.%Revisar explanation of nonlinear adaprive meshing
Specially, for larger indentation cases, this option shows a more stable model with a 
good mesh convergence analysis.

A force-displacement curve, shown in Fig... is generated from the first assumption, 
for this case 
%Additionally due to these factors, the converge analysis   

For both cases 

\subsection{Verification of the Simulation Model}

\subsubsection*{Mesh Convergence Analysis}
\subsubsection*{Platform vs Fixed Support}


%----------------------------------------------------------------------------------
\section{Nearby point}


