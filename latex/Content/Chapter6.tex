% Chapter 5

\chapter{Results} % Main chapter title
\label{resultschapter} % For referencing the chapter elsewhere, use \ref{Chapter1} 

%----------------------------------------------------------------------------------
\section{Overview and Analysis}
Upon applying the RSO and MATLAB optimization to identify the optimal material parameter set, 
multiple sets were calculated. 
Each simulation deployed various Neo-Hookean parameters sets and 
had a unique load-displacement solutions. These load-displacement curves were 
juxtaposed with the experimental data and the RMSE and NRMSE (Equation \ref{eq:rmse} and \ref{eq:nrmse}) was calculated to quantify the quality 
of each solution. To deepen the evaluation, additional performance metrics were incorporated: 
Mean Relative Error (MRE) (Equation \ref{eq:mre}), Mean Absolute Percentage Error (MAPE)
\begin{align}
    \text{MAPE} = \frac{100\%}{n} \sum_{i=1}^{n} \left| \frac{y_i - \hat{y}_i}{y_i} \right| \,,
\end{align}
and  Relative Root Mean Square Error (RRMSE)  
\begin{align}
    \text{RRMSE} = \sqrt{\frac{1}{n} \sum_{i=1}^{n} \left( \frac{y_i - \hat{y}_i}{y_i} \right)^2} \,.
\end{align}
%https://www.analyticsvidhya.com/blog/2021/10/evaluation-metric-for-regression-models/   
Performance metrics are critical for evaluating the predictive accuracy of the computational model.
These metrics provide insights into how well the model's prediction align with the 
experimental data. The combination of these metrics allowed for a more comprehensive 
assessmetn of all the simulations performance. For instance, the RMSE and NRMSE measured the
absolute deviation of the predicted values from the observed data, providing an overall 
measure of the computational model accuracy. On the other hand, metrics like MRE, MAPE. and 
RRMSE are useful when considering relative errors. These metrics weighed errors in relation
to the actual size of the actual values \cite{Rajagukguk2020}.\\

The simulation results that produced the lowest performance metrics were set aside as possible
optimal candidates. In particular, the paremeters with the lowest NRMSE, MRE and RMSRE were  
investigated. Moreover, a set with an elevated incompressibility parameter was chosen for 
analyzing its influence on validation cases.
The parameters that were determined to provide the best fit to the 
experimental data based on these assessments were listed on Table \ref{tab:materialsetbestfit}.
\begin{table}[h!]
    \centering
    \begin{tabular}{|c|c|c|c|c|c|c|}
    \hline
    Set & $\mu$ (Pa) & $D_1$ (MPa\textsuperscript{-1}) & NRMSE & MRE & RRMSE & Optimization Method\\
    \hline
    1 & 9999.7 & 5.8 & 0.0339 & 0.1098 & 0.2067 & RSO 2P RR\\
    2 & 9975.2 & 5.3 & 0.0340 & 0.1097 & 0.2069 & RSO 2P RR\\
    3 & 10200 & 1.1 & 0.0533 & 0.1139 & 0.1964 & MATLAB Poly 4 MRE\\
    4 & 12453 & 139.3 & 0.0648 & 0.1713 & 0.2592 & RSO 2P FR\\
    \hline
    \end{tabular}
    \caption[Best material parameter sets]{Neo-Hookean material parameter sets that demonstrated the best fit to the experimental data of the EM II.}
	\label{tab:materialsetbestfit}
\end{table}

\subsection*{Assessment of Performance Metrics}
Defining good performance metrics values required a multi-dimensional approach. 
Firstly, the load-displacement curves' visual inspection revealed a close approximation 
to the experimental data (Fig. \ref{fig:bestfitcandidatescurve}).

The load-displacements curves illustrated that the initial slope of EM II was steeper compared to 
all calculated candidate sets. As the displacement increased sets \SI{1}{} through \SI{3}{} intersected 
the experimental curve. Set \SI{1}{} and \SI{2}{} were nearly identical, initially exhibited an increasing 
slope, yet displayed a tendency to flatten towards the end of the curve, intersecting the experimental curve 
a second time.

On the other hand, set \SI{3}{} crossed the experimental curve around $u=\SI{1.5}{\milli \meter}$, showcasing 
a steeper slope relative to the experimental curve. 

Set \SI{4}{} displayed a pronounced curvature at the outset, followed by an increase in steepness of its slope.
However, it remained constantly below throughout the experimental data.\\

%load-displacements results curves comparison
\begin{figure}%
    \centering
   \quad
    \begin{tikzpicture}[scale=1]
        \begin{axis}[
            xmax=4.2,xmin=0,
            ymin= 0,ymax=0.6,
            ytick={0,0.1,0.2,...,0.5},
            xlabel={Displacement $u [mm]$},
            ylabel={Force reaction $F_{II} [N]$},
            grid = major,
            legend pos= north west]
            \addplot+[smooth, no markers, thick] table [y=$Force$, x=Def]{Table/RSO/expdatatop.dat};
            \addplot+[smooth, no markers, thick] table [y=$Force$, x=Def]{Table/results/set199997_58.dat};
			\addplot+[smooth, no markers, thick] table [y=$Force$, x=Def]{Table/results/set299752_53.dat};
            \addplot+[smooth, no markers, thick] table [y=$Force$, x=Def]{Table/results/set310200_11.dat};
            \addplot+[smooth, orange, no markers, thick] table [y=$Force$, x=Def]{Table/results/set412453_1393.dat};
            \legend{EM II-MP,Set 1,Set 2, Set 3, Set 4}
        \end{axis}
    \end{tikzpicture}%
   \caption[Best material parameter sets load-displacement curves]{Visual analyis of the load-displacement curves of the best material parameter sets with the lowest NRMSE, MRE, and RRMSE and the experimental data.}%
   \label{fig:bestfitcandidatescurve}%
\end{figure}

In addition to the visual analysis, the mean of the performance metrics for all models were calculated and 
compared to its lowest value. Similarly, a baseline model of each performance metric was calculated and 
also compared to the candidate sets (Table \ref{tab:performancegoodness}). 

To clarify, the lowest value of the NRMSE was \SI{0.0339}{}. The mean NRMSE across all models was 
\SI{0.1662}{}, suggesting that models with an NRMSE below this value outperformed the average.
Consequently, the baseline NRMSE, represented a simple model predicting the mean load-displacement, was 
\SI{0.6835}{}, indicating that models with an NRMSE below this value surpassed the baseline model.\\ 

\begin{table}[h!]
    \centering
    \begin{tabular}{|>{\centering\arraybackslash}m{2cm}|>{\centering\arraybackslash}m{2cm}|>{\centering\arraybackslash}m{2cm}|>{\centering\arraybackslash}m{2cm}|}
    \hline
    Metric & Set values & Mean & Baseline \\
    \hline
    NRMSE &  0.0339 0.0340 0.0533 0.0648 & 0.1662 & 0.6835 \\
    \hline
    MRE &  0.1098 0.1097 0.1139 0.1713 & 0.2689 & 0.5916\\
    \hline
    RRMSE & 0.2067 0.2069 0.1964 0.2592 & 0.2688 & 0.6835\\
    \hline
    \end{tabular}
    \caption[Goodness of fit]{Comparison of the perfomance metric of each candidate set with their mean value and baseline model value, extracted from all calculated simulations.}
	\label{tab:performancegoodness}
\end{table}

Given the proximity of the values among all candidates, except for set \SI{4}{}, it was 
initially inferred that utilizing NRMSE as the sole performance metric would be adequate for 
evaluating the goodness of fit in this specific case, an indentation of $h=\SI{4}{\milli \meter}$
at the center of the surface.

\section{Validation of Computational Model}

\subsection{Deeper Indentation}

\subsection{Nearby Point - Blue Point}

\subsection{Deformation Profile Analysis}

\section{Framework Proposal}

\section{Limitations and Implications of the Results}



%\subsection{First Experimental model}
%The chosen experimental technique for the inverse identification for this project was indentation. 
%The test specimen used for this experiment was a ultra-soft polyurethane resin. 
%As shown in Fig.. the specimen possesses a ellipsoidal form with 
%with a minor radius \(r_1\) = 35 mm and a major radius \(r_2\) = 60 mm. This was positioned
% on a fixed platform that suited the ellipsoidal geometry of the 
% specimen to constrain its movement. 
% The specimen was tested in a indentation test configuration with a tensile/compression machine.
% To achieve this congiguration a pin with a rounded head made of structural steel, 
% with a radius of \(r_3\) = 3 mm was attached 
% to the holding grips followed by a force load cell. 
%The result of indentation test was a load-displacement points. The approximated 
%polynomial curve was used as a reference for the material modeling.

%test specimen is loaded at a quasi-static rate

% The measured force reation \(F_1\) data showed a very small number, so the 
% first 50 N load cell displayed a lot of noise in the measurements. 
% Therefore, the load cell was change to 10N to reduce this interference. 
%The 10 N load cell displayed the force-displacement curve of the indenter and the specimen
% in a finer way. Furthermore, in order to get the measurement of the load and 
% unloading process of
% the indentation a displacement sensor was attached to the tensile machine.

% The indentation depth \(h_1\) selected for the first model was 3,8 mm on the middle of the 
% top surface of the specimen. This indentation depth surpasses the pin radius \(r_3\) and 
% was chose arbitriarily to analyze the behavior of the material on the defined position.
 %reference?
 Additionally, it was observed that in soft materials it is easier to capture 
 some parameters with a larger indentation. Some references also observe that with
 indentation depth lower than indenter radius has a lot of noise and do not describe
 th results accurately. %reference?

%\begin{figure}[th]
%    \centering
%    \begin{tikzpicture}
%        %\pgfplotsset{%legend outside the plot
%        %every axis legend/.append style={ at={(1.05,0.95)}, anchor=north west,legend columns = 1}}
%        \begin{axis}[
%            %axis lines=middle,
%            %x label style={at={(axis description cs:0.5,-0.1)},anchor=north},
%            %y label style={at={(axis description cs:-0.1,.5)},rotate=90,anchor=south},
%            xlabel={Displacement $u [mm]$},
%            ylabel={Force reaction in Z-Axis $F_z [N]$},
%            legend pos= north west]
%            
%            \addplot+[smooth, mark size = 1pt] table [y=$Force$, x=Def]{Table/data1.dat};
%            %\addplot+[smooth] table [y=Force, x=$Def$]{Table/data2.dat};
%            \legend{Experimental data}%,$l_2$}
%        \end{axis}
%    \end{tikzpicture}
%    \caption[Expdata]{Experimental Load-displacement curve.}
%    \label{fig:testgraph2}
%\end{figure}

%\subsection{Second Experimental model}
%The second experimental model was developed by Yokohama National University. Similar to 
%the first experimental model the test specimen and the platform were it lies, has the 
%same dimensions, minor radius $r_1 = \SI{35}{\milli \m}$ and a major radius \(r_2\) of 60 mm. The
%test specimen is also made from the same material, ultra-soft polyurethane resin.
%
%The indenter on the other hand, is a sphere made of ruby, the sphere radius is also 
%equal to the radius of the pin \(r_s\) 3 mm and attached to it, is the force load cell.
%
%A laser is used to measure the displacement which results in a load-displacement curve.
%With this model it is possible to not only determine the toal force reaction, but also
%it's components \(F_x\),  \(F_y\) and \(F_z\). Furthermore, with the laser it is also
%possible to observe the deformation not only in one point but around the whole area. 
%This allows as to analyze the deformation of the whole structure.
%
%The indentation speed selcted was % ask for which speeed
%and with an indentation depth of \(h_s\) is 4 mm. With this experiment, 4 key points on 
%the sepcimen's surface were chosen: First, in the middle and three other points, one to right, 
%one down, and one diagonal to middle, forming a square with a distance between points 
%of \(d_s\) 20 mm. %add figure with points

%----------------------------------------------------------------------------------
%\section{Material model framework assumptions}
%
%The first point to be analyzed, which is used to build a material model is point No. 1,
%in the middle of the surface. The advantages from this case, is the less influence of
%external factors. For this case it is vaiable to assumed, that shear stresses can be 
%neglected and offers a simple model to focus on the material definition.
%
%For this project, there is a focus on the limitation of each material model, 
%departing from an ideal scenario. From this point on the material will be build 
%accordingly and for each model the influence of the material parameters is going
%to be assessed.


%----------------------------------------------------------------------------------
\section{Material model}

%n an ideal and first scenario, this material can be assumed as linear, isotropic, 
%lastic and nearly imcompressible. For this case, there are two main variables, the Young's
%odulus \(E\), and the Poisson's ratio $\nu$.

%comentario sobre la influencia del bulk modulus y poissons ratio
From the parametric analysis, it is possible to see that the bulk 
modulus of this material does not possess a big impact in the FE 
simulation results. This conclusion combined with the results 
from the Poisson ratio in the first material model coincide with the 
statements from Bergström, where it is no vital to know these parameters 
to obtain accurate FE computational models, as these have limited
influence on the mechanical response. \cite{Bergström2015} %pag64Bergströom

%---------------------------------------------------------------------------------