% Chapter 6

\chapter{Conclusion and Outlook} % Main chapter title

\label{Chapter7} % For referencing the chapter elsewhere, use \ref{Chapter1} 

%----------------------------------------------------------------------------------
\section{Summary and Contributions}
The goal of this study was the identification of soft material parameters through 
an iFEM process. Furthermore, this work focused on developing a framework for identifying the 
essential material parameters of soft materials, particularly, those that experience 
large deformations and display nonlinear responses. The aim of this framework was the 
approximation of the behavior of such complex materials via a simplified material model. 
Moreover, this approach's purpose was to potentially apply this framework in medical research,
particularly for the identification of the material parameters of organs.\\

To fulfill this goal, a combination of both experimentation, performed primarily by Yokohama National University, 
and simulation was utilized. The experiments were conducted to collect load-displacement data from
a representative material sample, ultra-soft polyurethane. A computational model was then 
established, mirroring the physical experiment setup, and an initial guess of the material 
parameters was input.\\

In establishing the computational model, the primary challenge encountered was the 
high distortion effect of elements during simulation calculations. To address this issue, the nonlinear adaptive 
meshing option solution was applied, facilitating the verification of the designed 
computational model. Regarding the material model, varying the levels from linear elasticity to 
Neo-Hookean were explored. In this process, the key parameters were identified, and their interrelationships were analyzed.\\

A Response Surface Optimization (RSO) algorithm was employed to optimize these 
parameters, which were assessed by a performance metric defined as the difference between the 
experimental and simulated results. The RSO method allowed the identification and analysis of the 
key design parameters and their influence on the material's behavior. 
To enhance the RSO results, various performance metrics were calculated and examined for 
the top candidates. This evaluation was conducted to determine if the same material could be approximated 
with different sets of material parameters. Additionally, the approximated polynomial derived 
from the material parameter sets and their corresponding performance metric aided in evaluating the 
candidates yielded by the RSO, which confirmed the correctness of the setup.\\

Moreover, three validation cases were conducted to evaluate the material parameter identification 
approach's robustness. While these cases affirmed the effectiveness of this approach under 
specific conditions, it also revealed inherent limitations. These limitations served to 
lay out the boundaries of this approach's applicability, providing valuable insights for the 
optimization of this framework for future research.

\section{Recommendations for Future Research}
Future research should consider extending and improving the applicability of the developed 
framework beyond the specific condition validated in this work. Firstly, it is suggested to explore 
a multi-point indentation to capture the overall mechanical behavior of the material, rather than a 
single indentation. This study illustrated that the blue point validation case offered promise 
for evaluating the force reaction components. Therefore, shifting the focus from solely utilizing 
the most basic case, the middle point indentation, to including multiple points is recommended.\\

Additionally, the reassessment of the selection of experiments employed for the iFEM method is 
encouraged. The data collected from each experiment can provide different insights about the material, therefore, 
the selection of the right experiments significantly influences the understanding of the material's behavior and 
improves the accuracy of the generated models. This includes, expanding the depth of indentation or incorporating 
measurements of the deformation shape of the whole sample's surface.\\

Moreover, the constraints on the specimen were found to be relevant to the results of the computational model.
Thus, investigations into the friction value between the used platform and the specimen could 
provide more accurate values.\\

When considering the application of this study in organs, it is recommended that indentations from 
different directions be considered. The results indicated that the surface gradient impacts the force-displacement data the
further the indentation is moved from the middle point. As such, considering perpendicular indentations to the top 
of the surface, or indentations from the side, could offer comprehensive information about the organ behavior.\\

Furthermore, the exploration of other hyperelastic models should be evaluated. The Neo-Hookean material 
model could not perfectly match the curve, indicating that other material models, e.g., Mooney-Rivlin or Yeoh model, may 
offer better suitability and potentially capture the material's response more accurately. Another area to explore is the range
 of input parameters for the RSO for the selected material model. The results indicated that there may be room 
for optimization in the input range to provide even more precise results.\\

Lastly, future research should consider a methodology to establish the error metric threshold, particularly for studies
involving biomaterials. Collaborating with experts in the field can help determine the level of accuracy the model 
needs to achieve to be deemed suitable for practical applications.

\section{Conclusions and Final Remarks}

In conclusion, this study contributed to the progress of the mechanical characterization of soft tissues
by proposing a framework for identifying material parameters through an iFEM approach. This framework's
distinctive feature was the integration of a systematic hierarchy for material models, facilitating the 
attainment of the most precise material response, while maintaining the simplicity of the material model to the 
greatest extent possible.\\

However, the study acknowledged its limitations, in particularly for practical application. By considering
multi-point indentation, or reevaluating the indentation position, investigating the specimen constraints, 
exploring alternative material models, and redefining the error metric threshold, future work can build upon the 
foundation established in this study.\\

The potential applications of this proposed framework in medical research, particularly in the characterization 
of organs for \textit{ex vivo} experiments, underscored the importance of this work. The efforts towards 
the development of simple but accurate and efficient material models will continue to be an active area
of research, and we hope that our work serves as a stepping stone for future endeavors in this field. 
