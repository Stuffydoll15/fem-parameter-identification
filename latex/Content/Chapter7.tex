% Chapter 6

\chapter{Conclusion and Outlook} % Main chapter title

\label{Chapter7} % For referencing the chapter elsewhere, use \ref{Chapter1} 

%----------------------------------------------------------------------------------
\section{Summary and Contributions}
The goal of this study was the identification of soft material parameters through 
an iFEM process. Furthermore, this work focused on developing a framework for identifying the 
essential material parameters of soft materials, particularly, those that experience 
large deformations and display nonlinear responses. Through this framework the 
approximation of the behavior of such complex materials via a simplified material model
was aimed. Moreover, this approach's purpose was to potentially applied this framework in medical research,
particularly for the identification of the material parameters of organs.\\

To fulfill this goal, a combination of both experimentation, performed primarily by Yokohama National University, 
and simulation was utilized. The experiments were conducted to collect load-displacement data from
a representative material sample, ultra-soft polyurethane. A computational model was then 
established, mirroring the physical experiment setup, and a initial guess of the material 
parameters was inputted.\\

In establishing the computational model, the primarily challenge encountered was the 
high distortion effect pf elements during simulation calculations. To address this issue, the nonlinear adaptive 
meshing option solution was applied, facilitating the verification of the designed 
computational model.\\

A Response Surface Optimization (RSO) algorithm was employed to optimize these 
parameters, which were assessed by a performance metric defined as the difference between the 
experimental and simulated results.The RSO method allowed the identification and analysis of the 
key design parameters and their influence in the material's behavior. 
To further enhance the RSO results different performance metrics were calculated and analyzed for 
the best candidates to evaluate if the same material can be approximated with different material 
parameter sets.\\



%In the first chapter, the significant issue of biomechanical characterization of soft tissues, 
%which is relevant in medical research, was addressed. Moreover, it was mentioned while traditional methods rely on experts 
%assumptions and experience, they lack quantifiable data necessary for computer-assisted systems.
%This highlights 

\section{Recommendations for Future Research}


\section{Conclusions and Final Remarks}



