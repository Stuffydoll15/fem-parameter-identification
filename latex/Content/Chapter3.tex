% Chapter 2

\chapter{Inverse Finite Element Method for Material Parameter Identification} % Main chapter title

\label{Chapter3} % For referencing the chapter elsewhere, use \ref{Chapter1} 

%----------------------------------------------------------------------------------
\section{Experimental model}

The chosen experimental technique was indentation. In Fig.. the ellipsoidal specimen
with the a first radius \(r_1\) of 30 mm and a second radius \(r_2\) of 60 mm is positioned
 in a rest status(?) on a fixed platform that suits the ellipsoidal geometry of the 
 specimen. A metal pin with a rounded head with a radius \(r_3\) 3 mm is attached 
 by the holding grips followed by a force load cell. 
 The measured force data showed a very small number, so the 
 first 50 N load cell displayed a lot of noise in the measured data. 
 Therefore, the load cell was change to 10N to reduce this interference. 
The 10 N load cell displayed the initial contact between the indenter and the specimen
 in a finer way. In order to get the measurement of the load and unloading process of
 the indentation a displacement sensor was attached to the tensile machine

%----------------------------------------------------------------------------------
\section{Material model framework assumptions}


%----------------------------------------------------------------------------------
\section{Computational model}
The quasi static nature of the indentation experiment allows the use of a static 
structural analysis.

Their a two main factors which increases the complexity of the validation of the simulation
and those are, the contact nonlinearity, and the element distortion due to indentation
 experiment. These issues make the computational time expensive, as it requires to manual 
 solutions for the meshing in the area of importance, and small time steps.

A force-displacement curve, shown in Fig... is generated from the first assumption, 
for this case 
%Additionally due to these factors, the converge analysis   

For both cases 
%----------------------------------------------------------------------------------
\section{Material model}

In an ideal and first scenario, this material can be assumed as linear, isotropic, 
elastic and nearly imcompressible. For this case, there are two main variables, the Young's
Modulus \(E\), and the Poisson's ratio $\nu$.

%---------------------------------------------------------------------------------