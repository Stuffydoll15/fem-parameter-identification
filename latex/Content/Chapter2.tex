% Chapter 2

\chapter{Introduction} % Main chapter title

\label{Chapter2} % For referencing the chapter elsewhere, use \ref{Chapter1} 

Precise knowledge about the biomechanical characterization of soft tissues has received attention
in medical research, e.g., medical image analysis and visualization.
For many years, the obtained medical diagnosis have come from assumptions of experts or 
accumulated experience. This information, although proven to be useful, has its limitations 
when computed-assisted systems like, medical diagnosis, therapy, and training, rely on 
more quantifiable data\cite{Kauer2002}. To gather this data, it is important to gain access
to the tissues and perform in-vivo testing experiments. For organs, this procedure is nearly 
impossible to achieve, due to the involvement of an invasive procedure, and the lack of constant
and reproducible external and internal factors. \\ 

One of the complications associated with the extraction of the organ is that some material 
properties may change despite examination of the same organ. Their biomechanical properties 
depend on other factors, such as changes in blood pressure, changes in material properties 
over time, symptoms from diseases, etc. Furthermore, another encountered issue is the lack 
of replication, due to the use of different individuals organs, which includes more external
 factors to add to the equation. Moreover, given a tissue sample, it is difficult to 
 characterize the organ's material properly due to its anisotropy property. The material 
 properties change, resulting in an inaccurate result.
%not only due to the lack of a tissue sample, but also because the time-dependent material properties. 
\\

In the situation where the soft material's data can be gather in a constant, fast and reliable
process, the data enables the system to predict the behavior of soft tissues and give pre-operative
 calculations. This shows that material models represent a vital part for medical research, 
 specially for the use of computational models, as their help to increase the accuracy of the 
 simulation and its applications in other systems. 

In soft materials analyses, a nonlinear situation is mostly encountered, for which a finite 
element method becomes a common approach. The application of the finite element method
 facilitates the analysis of complex structures with complex material behavior, and aids in 
 solving of continuum mechanical problems. Nevertheless, in order to simulate a material with 
 this complexity also requires complex algorithms with high computational costs. \\
 
One of the goals of this study to identify the key parameters of the soft materials, and their 
influence in the construction of a material model. These key parameters the attempt to 
 approximate such a complex material will be done, and this simplified material model will be 
 validated for its future applications in medical research.\\
 
 With the application of the experiment testing, and the finite element method it is possible 
 to identify some key material parameters through an inverse finite element method approach. With this 
 method a framework can be established and the results of computational model can be matched 
 to the experimental data, and afterwards be validated with other experiments.

%----------------------------------------------------------------------------------
\section{State of the art}

\subsection{Experimental techniques for soft materials}

%Soft materials challenges in experimental deign

- Soft synthetics materials like soft gels are one example for soft synthetic materials. These
are commomly applied for tissue engineering applications. Nevertheless, due to their elastic 
modulus range (kPa) present some challenges for the design of experimental testing\cite{Liu2009}.

\subsubsection{Uniaxial testing}
This method allows the validation of several computational models as it provides with
 searched parameters done with other experimental procedure. 

\subsubsection{Aspiration experiment}

Tissue aspiration experiments introduces an aspiration tube which is put against the 
soft tissue, generating a vacuum. An advantageous feature of this experiment is that 
it can be perfomed in-vivo and ex-vivo.
With the help of a mirror placed next to aspiration 
hole, the reflection of the side-view of the tissue can be captured with a video camera.
This camera captures the images of the iluminated surface of the material and the 
aspiration pressure is captured through a sensor. Through this process the captured 
profile of the tissue is obtained and this can be used to characterize the deformation 
and analyze the viscoelastic properties of the soft tissue\cite{Kauer2002}.

\subsubsection{Indentation}
Indentation have being gaining popularity in the last decades and it is now one of
 the most spread experiments for material parameter identification.(?)

Indentation possesses advantageous characteristics for the mechanical characterization
of soft materials. \cite{Liu2009}

As some materials do not allow the use of uniaxial or biaxial tensile testing, the
 use of identation testing is essential for this case. 

-identation in materials
- Identation in soft materials (organs)
- Why is identation relevant in organs
-what advantages and disadvantges does identation provides
-why is relevant for this project 

%---------------------------------------------------------------------------------
\subsection{Material Modeling of Soft Tissues}
 
Relevant papers tends to use hyperelastic models to represent soft materials as the viscoelasticity 
is usually neglected. 
% why do they use hyperelastic formulas? Why is it possible to neglect the viscoelasticity

%viscoelastic example
For example, in the Aspiration experiment conducted by Kauer the material they used to model the 
polymer established in the experiment is:

As derivative from this formula for the uteri metrial modeling the formula used was:

%---------------------------------------------------------------------------------
\subsection{Inverse Finite Element Method for Parameter Identification}

An inverse finite element (FE) approach requires usually a certain experimental model,
 which generates certain information e. g. load-displacement curve, and through 
a verified computational model match the given data curve to obtain further information 
of the material's behavior e. g., stress-strain curve.

Specially for nonlinear cases \cite{Husain2004}, where the complexity of the problems 
increases, and the interest is focused to generate an action which results in a 
certain output response, is where an inverse finite element approach can be helpful 
to discover a certain variable going from an ouput data. Through an iterative process it is 
possible to describe the material's behavior and validate the output data it 
through other established testing e. g., uniaxial testing.

Though this approach does not always give a hundred percent match in all obtain points 
or zones, it allows the researcher to understand the influences of certain parameters 
for the materials. This is specially useful for complex materials as biomaterials. 

Biomaterials, as mentioned previously, depends on multiple external factors, e.g., blood 
pressure, affected diseases and the their material properties is constantly changing. 
This issue does not allow the researcher to develop a proper material model which is 
usable for multiple use-cases. 

Therefore, the importance of the inverse element method as relevant key for estimating 
constantly changing parameters in soft materials.

For biomechanical models, where the models require knowledge from local properties \cite{Chai2013},
as the biomaterial is not isotropic; it is possible to identify a parameter e.g. Young's Modulus 
from a 3D model. The model can be matched to multiple experiments and multiple samples in different areas,
which allows a better representation of the material for further analysis.

The inverse FE approach can used by optimizing the searched parameter by matching the simulated data
to a section of a experimental curve and extending this process through some iterations. 
Nevertheless, it is important to clarify that this method also requires making assumptions to some values.
Furthermore, it is relevant to document these assumptions for the further analysis. 
With the combination of assumptions, experimental data, and a optimized and matched simulation curve, it
is possible to solve the complexity of biomechanical models.
 
In next sections some of the experimental models and the material models for bio and soft materials are 
going to be explained to get a further understanding in how is possible to get a realiable computational 
model for further reasearch

\subsubsection{Synthetic soft materials}

Synthetics materials are commonly used to validate an inverse parameter identification process. 
Usually these synthetic, soft materials provide similar mechanical behavior to it's biomaterials 
counterparts. This characterization allows to validate a proposed inverse finite element approach process
before its applicatoin with a biomaterial, where the measurements to gather the experimental data are 
some in-vivo, and more challenging to recreate.


For example, Silgel, a very soft gel-like material \cite{Kauer2002} was used for the experimental 
validation of the inverse finite element method proposed, to characterized the tissue of 
a human uteri. In this work, the tensile behavior of the material was predicted through the 
parameters obtained in the aspiration method. The matching procedure is optimize through 
an objective function, which consists of the squared differences between the simulation 
and exprimental data. With an optimization algorithm an optimal set of the following parameters 
was found: the material parameters \(\mu_i\) [N/m\textsuperscript{2}] and the bulk Modulus
\(\kappa\) [N/m\textsuperscript{2}]. This method showed good prediction quality of the mentioned 
material parameters.

\subsubsection{Biomaterials}
Biomaterials as mentioned before, represent a challenge due its difficult access and lesser replicability.
Therefore these materials are usually used for the experimetnal validation of a methd applied previously in 
synthetic materials. 
 Following the first example of the Silgel in the previous section, the inverse finite element parameter
 estimation is applied now on human uteri \cite{Kauer2002} through in vivo and ex vivo measurements of the human tissue of 
 different patients. It was mentioned, that in comparison from the silgel the uterus possesses a complex 
 multi layered structure with strongly anistropic and viscoelastic properties. Nevertheless, five 
 material parameters were determined, based on the strain energy function to model a human uterus (Yamada 1970).
Through the same inverse method applied with the synthetic material, the obtained parameters facilitated the prediction of stress-elongation curves for tensile experiments. The 
 resulting curves showed the difference of stiffness for in vivo and ex vivo measurements and the material 
 singurality for each uterus.



\subsection{Standard Verification and Validation for computational solid mechanics (ASME)}
\subsubsection{VV40}


%---------------------------------------------------------------------------------