% Chapter 2

\chapter{Introduction} % Main chapter title

\label{Chapter2} % For referencing the chapter elsewhere, use \ref{Chapter1} 

%----------------------------------------------------------------------------------
\section{State of the art}

\subsection{Experimental techniques for soft materials}
\subsubsection{Uniaxial testing}

\subsubsection{Indentation}
Indentation have being gaining popularity in the last decades and it is now one of
 the most spread experiments for material parameter identification.

As some materials do not allow the use of uniaxial or biaxial tensile testing, the
 use of identation testing is essential for this case. 

-identation in materials
- Identation in soft materials (organs)
- Why is identation relevant in organs
-what advantages and disadvantges does identation provides
-why is relevant for this project 

%---------------------------------------------------------------------------------

\subsection{Inverse Finite Element Method for Parameter Identification}

An inverse finite element (FE) approach requires usually a certain experimental model,
 which generates certain information e. g. load-displacement curve, and through 
a verified computational model match the given data curve to obtain further information 
of the material's behavior e. g., stress-strain curve.

Specially for nonlinear cases (husain2004), where the complexity of the problems 
increases, and the interest is focused to generate an action which results in a 
certain output response, is where an inverse finite element approach can be helpful 
to discover a certain variable going from an ouput data. Through an iterative process it is 
possible to describe the material's behavior and validate the output data it 
through other established testing e. g., uniaxial testing.

Though this approach does not always give a hundred percent match in all obtain points 
or zones, it allows the researcher to understand the influences of certain parameters 
for the materials. This is specially useful for complex materials as biomaterials. 

Biomaterials, as mentioned previously, depends on multiple external factors, e.g., blood 
pressure, affected diseases and the their material properties is constantly changing. 
This issue does not allow the researcher to develop a proper material model which is 
usable for multiple use-cases. 

Therefore, the importance of the inverse element method as relevant key for estimating 
constantly changing parameters in soft materials.

For biomechanical models, where the models require knowledge from local properties (chai c2013),
as the biomaterial is not isotropic; it is possible to identify a parameter e.g. Young's Modulus 
from a 3D model. The model can be matched to multiple experiments and multiple samples in different areas,
which allows a better representation of the material for further analysis.

The inverse FE approach can used by optimizing the searched parameter by matching the simulated data
to a section of a experimental curve and extending this process through some iterations. 
Nevertheless, it is important to clarify that this method also requires making assumptions to some values.
Furthermore, it is relevant to document these assumptions for the further analysis. 
With the combination of assumptions, experimental data, and a optimized and matched simulation curve, it
is possible to solve the complexity of biomechanical models.
 
In next sections some of the experimental models and the material models for bio and soft materials are 
going to be explained to get a further understanding in how is possible to get a realiable computational 
model for further reasearch

\subsubsection{Biomaterials}

\subsubsection{Hydrogels}

\subsection{Standard Verification and Validation for computational solid mechanics (ASME)}
\subsubsection{VV40}


%---------------------------------------------------------------------------------