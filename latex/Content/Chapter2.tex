% Chapter 2

\chapter{State of the Art} % Main chapter title

\label{Chapter2} % For referencing the chapter elsewhere, use \ref{Chapter1} 

%----------------------------------------------------------------------------------
\section{Inverse Finite Element Method for parameter estimation}

Specially for nonlinear cases (husain2004), where the complexity of the problems increases, is 
where an inverse finite element approach can be helpful to discover a certain variable 
going from an ouput data.


\section{Material Experiment testing in Organs}

Indentation have being gaining popularity in the last decades and it is now one of
 the most spread experiments for material parameter identification.

As some materials do not allow the use of uniaxial or biaxial tensile testing, the
 use of identation testing is essential for this case. 

-identation in materials
- Identation in soft materials (organs)
- Why is identation relevant in organs
-what advantages and disadvantges does identation provides
-why is relevant for this project 

%---------------------------------------------------------------------------------