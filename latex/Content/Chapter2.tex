% Chapter 2

\chapter{Introduction} % Main chapter title

\label{Chapter2} % For referencing the chapter elsewhere, use \ref{Chapter1} 

Precise knowledge about the biomechanical characterization of soft tissues has gained 
attention in medical research, e.g. medical image analysis and visualization.
For many years, the obtained medical diagnosis have come from assumptions of experts or 
accumulated experience. This information although proven to be useful, has its limitations 
when computed-assisted systems like, medical diagnosis, therapy ,and training, rely on 
more quantificable data. (kauer 2002) To gather this data, it is important to gained access
to the tissues and perform in-vivo testing experiments, where in cases of organs it is nearly impossible to achieve, because this 
would involve a very invasive procedure. Also in case of extraction some of the properties
may change as in case of biomaterials. Their biomechanical properties also depends from other 
variables e.g., changes in blood presure. Furthermore, another encountered issue is the lack of samples, 
not only due to the lack of a tissue sample, but also because the time-dependent material 
properties. This will cause a change a in the material properties and analogically, an 
inaccurate result.\\

In the situation where the soft material's data can be gather in a constant, fast and realiable
process, the data enables the system to predict the behaviour of soft tissues and give pre-operative
 calculations. Therefore, the material models represent a vital part for the computational 
 models, as their help to increase the accuracy of the simulation and ist applications in other
 systems. 

In soft materials analyses a non-linear situation is mostly encountered, for which a finite 
element method becomes a common approach. The application of the finite element method
 facilitates to analyze complex structures with complex material behaviour and aids in 
 solving continuum mechanical problems. Nevertheless, in order to simulate a material with 
 complexity requires complex algorithms with high computational costs. One of the goals of 
 this study to identify the key parameters and their influence in a material model to 
 approximate such a complex material and applied a simplified material model for medical 
 research applications.\\
 
 With the application of the experiment testing ,and the finite element method is possible 
 to identify some key parameters through an inverse finite element method approach. With this 
 method a framework can be established and the results of computaional model can be matched 
 to the experimental data, and be validated with other experiments.

%----------------------------------------------------------------------------------
\section{State of the art}

\subsection{Experimental techniques for soft materials}
\subsubsection{Uniaxial testing}
This method allows the validation of several computational models as it provides with
 searched parameters done with other experimental procedure

\subsubsection{Aspiration experiment}

Tissue aspiration experiments introduces an aspiration tube which is put against the 
soft tissue, generating a vacuum. With the help of a mirror placed next to aspiration 
hole, the reflection of the side-view of the tissue can be captured with a video camera.
This camera captures the images of the iluminated surface of the material and the 
aspiration pressure is captured through a sensor. Through this process the captured 
profile of the tissue is obtained and this can be used to characterize the deformation 
and analyze the viscoelastic properties of the soft tissue. (kauer 2002)

\subsubsection{Indentation}
Indentation have being gaining popularity in the last decades and it is now one of
 the most spread experiments for material parameter identification.

As some materials do not allow the use of uniaxial or biaxial tensile testing, the
 use of identation testing is essential for this case. 

-identation in materials
- Identation in soft materials (organs)
- Why is identation relevant in organs
-what advantages and disadvantges does identation provides
-why is relevant for this project 

%---------------------------------------------------------------------------------
\subsection{Material Modeling of Soft Tissues}
 


%---------------------------------------------------------------------------------
\subsection{Inverse Finite Element Method for Parameter Identification}

An inverse finite element (FE) approach requires usually a certain experimental model,
 which generates certain information e. g. load-displacement curve, and through 
a verified computational model match the given data curve to obtain further information 
of the material's behavior e. g., stress-strain curve.

Specially for nonlinear cases (husain2004), where the complexity of the problems 
increases, and the interest is focused to generate an action which results in a 
certain output response, is where an inverse finite element approach can be helpful 
to discover a certain variable going from an ouput data. Through an iterative process it is 
possible to describe the material's behavior and validate the output data it 
through other established testing e. g., uniaxial testing.

Though this approach does not always give a hundred percent match in all obtain points 
or zones, it allows the researcher to understand the influences of certain parameters 
for the materials. This is specially useful for complex materials as biomaterials. 

Biomaterials, as mentioned previously, depends on multiple external factors, e.g., blood 
pressure, affected diseases and the their material properties is constantly changing. 
This issue does not allow the researcher to develop a proper material model which is 
usable for multiple use-cases. 

Therefore, the importance of the inverse element method as relevant key for estimating 
constantly changing parameters in soft materials.

For biomechanical models, where the models require knowledge from local properties (chai 2013),
as the biomaterial is not isotropic; it is possible to identify a parameter e.g. Young's Modulus 
from a 3D model. The model can be matched to multiple experiments and multiple samples in different areas,
which allows a better representation of the material for further analysis.

The inverse FE approach can used by optimizing the searched parameter by matching the simulated data
to a section of a experimental curve and extending this process through some iterations. 
Nevertheless, it is important to clarify that this method also requires making assumptions to some values.
Furthermore, it is relevant to document these assumptions for the further analysis. 
With the combination of assumptions, experimental data, and a optimized and matched simulation curve, it
is possible to solve the complexity of biomechanical models.
 
In next sections some of the experimental models and the material models for bio and soft materials are 
going to be explained to get a further understanding in how is possible to get a realiable computational 
model for further reasearch

\subsubsection{Synthetic soft materials}

Synthetics materials are commonly used to validate an inverse parameter identification process. 
Usually these synthetic, soft materials provide similar mechanical behaviour to it's biomaterials 
counterparts. This characterization allows to validate a proposed inverse finite element approach process
before its applicatoin with a biomaterial, where the measurements to gather the experimental data are 
some in-vivo, and more challenging to recreate.

For example, Silgel, a very soft gel-like material (M. Kauer, 2002) was used for the experimental 
validation of the inverse method proposed to characterized the tissue of of a human uteri. In this 
work, the tensile behaviour of the material was predicted through the parameters obtained in the 
aspiration method. 
\subsubsection{Biomaterials}




\subsection{Standard Verification and Validation for computational solid mechanics (ASME)}
\subsubsection{VV40}


%---------------------------------------------------------------------------------