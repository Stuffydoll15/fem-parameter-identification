% Chapter 4

\chapter{Inverse Finite Element Method for Material Parameter Identification} % Main chapter title
\label{iFEMthesis} % For referencing the chapter elsewhere, use \ref{Chapter1} 

The verification of the computational model in the previous chapter was followed by the implementation
of the Inverse Finite Element Method (iFEM) for material parameter identification.
As outlined in Chapter \ref{chapter:computationalmodel}, the Neo-Hookean material model
was employed due to its ability to describe the behavior of ultra-soft polyurethane over 
the range of the considered deformations. Moreover, in Subsection \ref{subsection:inverseFEMtheory} 
the iFEM importance for identifying material parameters for soft materials was explained.\\

In this chapter the process of parameter identifaction 
using this technique will be detailed. Firstly, the initial parameter estimation process will be discussed, 
followed by the optimization process. This revolved around iteratively refining the material parameters
to achieve the best match to the experimental load-displacement data of EM II. The two methods 
explored were ANSYS Response Surface Optimization (RSO) and a custom MATLAB routine. Lastly, the implementation 
and development of these optimization strategies, and their strengths and weaknesses, 
will be discussed. 

%----------------------------------------------------------------------------------
\section{Response Surface Optimization}
The Response Surface Optimization is a technique used in ANSYS for optimizing a design by creating 
a response surface, which represents the relationship between the design variables and the objective 
function or performance criteria. The objective of this feature is to find the optimal set of design 
variables that maximize or minimize the objective function.

The process to find parameters using the RSO in ANSYS generally involves the following steps:
\begin{enumerate}
    \item Define the design variables: The variables that influence the design are identified, e.g., geometric parameter, material properties.
    \item Define objective function: The parameter or objective which is going to be maximize or minimize is defined, e.g., stress, force reaction, volume. 
    \item Define constraints: Constraints or limitations are specified, such as lower or upper bounds.
    \item Generate Design of Experiments (DOE): Set of sample points are generated by varying the design variables. The DOE aims set a design a space efficiently to capture the relationships between the input and output parameters. Simulations are then performed for each set of design variables in the DOE.  
    \item Create a Response Surface: Based on the results of the simulations, the software constructs a surface by interpolating the discrete sampling points from DOE.
    \item Optimize the design: The optimal set of design variables are searched with optimization algorithms by iteratively evaluating the response surface and adjusting the design variables. The default optimization algorithm in ANSYS is the Multi-Objective Genetic Algorithm (MOGA), which supports multiple objectives and constraints. This algorithm aims to find a global optimum for the given problem \cite{Grebenisan2017}.
    \item Verify candidate points: The given candidate points fro the optimization tool are verified to observe if these satisfy the constraints and meet the target criteria.
\end{enumerate}



%\section{Material Modeling}
%In an ideal and first scenario, this material can be assumed as linear, isotropic, 
%elastic and nearly imcompressible. For this case, there are two main variables, the Young's
%Modulus \(E\), and the Poisson's ratio $\nu$.
%
%%comentario sobre la influencia del bulk modulus y poissons ratio
%From the parametric analysis, it is possible to see that the bulk 
%modulus of this material does not possess a big impact in the FE 
%simulation results. This conclusion combined with the results 
%from the Poisson ratio in the first material model coincide with the 
%statements from Bergström, where it is no vital to know these parameters 
%to obtain accurate FE computational models, as these have limited
%influence on the mechanical response. \cite{Bergström2015} %pag64Bergströom




%----------------------------------------------------------------------------------
\subsection{Objective Function Optimzation}

\subsection{Analysis and Comparison of Each Approach}

%---------------------------------------------------------------------------------
