% Chapter 4

\chapter{Inverse Finite Element Method for Material Parameter Identification} % Main chapter title

\label{Chapter5} % For referencing the chapter elsewhere, use \ref{Chapter1} 

\section{Procedure of IFEM}
%----------------------------------------------------------------------------------
\section{Material Modeling}

In an ideal and first scenario, this material can be assumed as linear, isotropic, 
elastic and nearly imcompressible. For this case, there are two main variables, the Young's
Modulus \(E\), and the Poisson's ratio $\nu$.

%comentario sobre la influencia del bulk modulus y poissons ratio
From the parametric analysis, it is possible to see that the bulk 
modulus of this material does not possess a big impact in the FE 
simulation results. This conclusion combined with the results 
from the Poisson ratio in the first material model coincide with the 
statements from Bergström, where it is no vital to know these parameters 
to obtain accurate FE computational models, as these have limited
influence on the mechanical response. \cite{Bergström2015} %pag64Bergströom

\subsection{Response Surface Optimization}

\subsubsection*{Linear Elastic Model}

\subsubsection*{Hyperelastic Model: Neo-Hookean}



%----------------------------------------------------------------------------------
\subsection{Objective Function Optimzation}

\subsection{Analysis and Comparison of Each Approach}

%---------------------------------------------------------------------------------
